\documentclass[a4paper,slidestop,xcolor=pst,dvips,blue]{beamer}

\input{slidesHeader.tex}

\newcommand{\command}[1]{$\backslash$#1\{\dots\}}

\title[\LaTeX]{Edición de Textos con \LaTeX}

\author[P. Sánchez]{\alert{Pablo Sánchez}}

\institute[ISTR]{
		   Dpto. Ingeniería Informática y Electrónica \\
		   Universidad de Cantabria \\
		   Santander (Cantabria, España) \\
		   \texttt{p.sanchez@unican.es}
}

\date{}

\begin{document}

\begin{frame}[c]
	\titlepage
	\begin{columns}
		\column{.5\linewidth}
			\centering \includegraphics[width=.33\textwidth,keepaspectratio=true]{images/istr.eps}
		\column{.5\linewidth}
			\centering
			\includegraphics[width=.25\textwidth,keepaspectratio=true]{images/uc.eps}
	\end{columns}
\end{frame}

\section{Introducción}

\subsection{Objetivos}

\begin{frame}[c]
	\frametitle{Objetivo}
	\begin{block}{Objetivo}
		Familiarizarse superficialmente con la edición de textos en \LaTeX, de manera que el alumno conozca el funcionamiento básico de \LaTeX, pierda el miedo a su utilización y pueda considerarlo como una opción viable para la elaboración de textos en el futuro.
	\end{block}
\end{frame}

\subsection{Bibliografía}

\begin{frame}[c]
	\frametitle{Bibliografía}
    \begin{thebibliography}{Lam94}

    \bibitem[Lam94]{lamport:1994}
    Leslie Lamport.
    \newblock {\em \LaTeX: A Document Preparation System}.
    \newblock Addison-Wesley Professional, 2 edition, Julio 1994.

    \end{thebibliography}
\end{frame}

\subsection{Conceptos Básicos}

\begin{frame}[c]
    \frametitle{¿Qué es \LaTeX?}
    \begin{enumerate}[<+->]
        \item Sistema de edición de documentos de textos.
        \item No \emph{WYSIWYG (``What You See Is What You Get'')}.
        \item Los documentos son texto plano donde el formato se especifica por medio de comandos y etiquetas (e.g. \emph{$\backslash$section\{Título\}}).
        \item Construido sobre \TeX.
        \item Basado en códigos de buenas prácticas internacionales.
        \item Extensible por medio de paquetes.
        \item Permite incorporar a los textos elementos complejos, como fórmulas \href{https://es.overleaf.com/learn/latex/Mathematical_expressions}{matemáticas}, \href{https://osl.ugr.es/CTAN/macros/generic/chemfig/chemfig-en.pdf}{moléculas} o
            \href{http://tug.ctan.org/info/latex4musicians/latex4musicians.pdf}{partituras}, entre otros, de manera textual (e.g. \texttt{$\backslash$int\_\{a\}\^{}\{b\} x\^{}\{2\}dx} es $\int_{a}^{b} x^{2} dx$).
        \item Muy utilizado a nivel académico.
    \end{enumerate}
\end{frame}

\begin{frame}[c]
    \frametitle{Elementos Básicos de un Documento \LaTeX}
    \begin{enumerate}[<+->]
        \item \command{documentclass}
        \item \command{begin}, \command{end}
        \item Preámbulo y documento.
        \item $\backslash$usepackage[utf8]\{inputenc\}
        \item \command{title}, \command{author}, \command{date}
    \end{enumerate}
\end{frame}

\begin{frame}[c]
    \frametitle{Formatos de Salida}
    \begin{description}[<+->]
        \item[DVI] Formato primitivo utilizado por \LaTeX.
        \item[Postscript] Formato independiente de la plataforma muy aceptado a nivel profesional. Soporta gráficos vectoriales (svg, eps).
        \item[PDF] Formato compacto de intercambio de documentos electrónicos, independiente de la plataforma. No soporta por defecto gráficos vectoriales.
    \end{description}
\end{frame}

\section{Comandos Básicos}

\subsection{Secciones y Niveles}

\begin{frame}[c]
    \frametitle{Secciones y Niveles}
    \begin{enumerate}
        \item \command{part}
        \item \command{chapter}
        \item \command{section}
        \item \command{subsection}
        \item \command{subsubsection}
        \item \command{paragraph}
        \item \command{subparagraph}
        \item \command{section*}
    \end{enumerate}
\end{frame}

\subsection{Adornos Básicos}

\subsection{Listas}

\section{Figuras, Tablas y Código}

\subsection{Figuras}

\subsection{Tablas}

\subsection{Listados de Código}

\section{Bibliografías e Índices}

\subsection{Bibliografías}

\subsection{Índices}

\section{Subdocumentos}

\section{Conclusiones}

\begin{frame}[c]
	\frametitle{Conclusiones}
	\begin{enumerate}[<+->]
		\item To be defined
	\end{enumerate}
\end{frame}

%\begin{frame}[c]
%	\frametitle{Bibliografía}
%    \nocite{lamport:1994}
%    \bibliographystyle{alpha}
%    \bibliography{latex}
%\end{frame}


\end{document}
