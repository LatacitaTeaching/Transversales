\documentclass[a4paper,slidestop,xcolor=pst,dvips,blue]{beamer}

\input{slidesHeader.tex}

\title[\LaTeX]{Edición de Textos con \LaTeX}

\author[P. Sánchez]{\alert{Pablo Sánchez}}

\institute[ISTR]{
		   Dpto. Ingeniería Informática y Electrónica \\
		   Universidad de Cantabria \\
		   Santander (Cantabria, España) \\
		   \texttt{p.sanchez@unican.es}
}

\date{}

\begin{document}

\begin{frame}[c]
	\titlepage
	\begin{columns}
		\column{.5\linewidth}
			\centering \includegraphics[width=.33\textwidth,keepaspectratio=true]{images/istr.eps}
		\column{.5\linewidth}
			\centering
			\includegraphics[width=.25\textwidth,keepaspectratio=true]{images/uc.eps}
	\end{columns}
\end{frame}

\section{Introducción}

\subsection{Objetivos}

\begin{frame}[c]
	\frametitle{Objetivo}
	\begin{block}{Objetivo}
		Familiarizarse superficialmente con la edición de textos en \LaTeX, de manera que el alumno conozca el funcionamiento básico de \LaTeX, pierda el miedo a su utilización y pueda considerarlo como una opción viable para la elaboración de textos en el futuro.
	\end{block}
\end{frame}

\subsection{Bibliografía}

\begin{frame}[c]
	\frametitle{Bibliografía}
     %% TODO(Completar)
\end{frame}

\subsection{Conceptos Básicos}

\begin{frame}
    \frametitle{¿Qué es \LaTeX?}
    %%
\end{frame}

\begin{frame}
    \frametitle{Ejemplos de Utilización de \LaTeX}
    %%
\end{frame}

\begin{frame}
    \frametitle{Elementos Básicos de un Documento \LaTeX}
    %%
\end{frame}

\begin{frame}
    \frametitle{Proceso de Generación de un Documento}
    %%
\end{frame}

\section{Comandos Básicos}

\subsection{Secciones y Niveles}

\subsection{Adornos Básicos}

\subsection{Listas}

\section{Figuras, Tablas y Código}

\subsection{Figuras}

\subsection{Tablas}

\subsection{Listados de Código}

\section{Bibliografías e Índices}

\subsection{Bibliografías}

\subsection{Índices}

\section{Subdocumentos}

\section{Conclusiones}

\begin{frame}[c]
	\frametitle{Conclusiones}
	\begin{enumerate}[<+->]
		\item To be defined
	\end{enumerate}
\end{frame}

\end{document}
